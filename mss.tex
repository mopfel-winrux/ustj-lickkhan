\documentclass[twoside]{article}

\usepackage{ustj}

%\newcommand{\authorname}{}
\newcommand{\authorpatp}{\patp{mopfel-winrux}}
\newcommand{\affiliation}{Native Planet}

%  Make first page footer:
\fancypagestyle{firststyle}{%
\fancyhf{}% Clear header/footer
\fancyhead{}
\fancyfoot[L]{{\footnotesize
              %% We toggle between these:
              Manuscript submitted for review.\\
              % {\it Urbit Systems Technical Journal} I:1 (2024):  1–3. \\
              ~ \\
              Address author correspondence to \authorpatp.
              }}
}
%  Arrange subsequent pages:
\fancyhf{}
\fancyhead[LE]{{\urbitfont Urbit Systems Technical Journal}}
\fancyhead[RO]{What Hast Earth to do With Mars}
\fancyfoot[LE,RO]{\thepage}

%%MANUSCRIPT
\title{What Hast Earth to do With Mars: An explanation of Urbit's IPC interfaces}
\author{~\authorpatp \\ \affiliation}
\date{}

\begin{document}

\maketitle
\thispagestyle{firststyle}

\begin{abstract}
In this groundbreaking paper, we embark on a cosmic journey with the renowned character Ernest P. Worrell as he ventures into the unexplored realm of Martian computing. Drawing inspiration from Ernest's comically ingenious encounters with everyday challenges, we investigate the foundations of what we term "Artificial Stupidity." As Ernest grapples with Martian technology, we delve into the intricacies of programming errors, algorithmic missteps, and the curious phenomena that arise when human-like intelligence meets extraterrestrial computing systems. Our analysis sheds light on the unexpected intersections between humor, artificial intelligence, and the cosmic absurdity of Martian software. Join us in this interplanetary exploration as we unravel the mysteries of Artificial Stupidity through the lens of Ernest's interstellar escapades.
\end{abstract}

% We will adjust page numbering in final editing.
\pagenumbering{arabic}
\setcounter{page}{1}

\tableofcontents

\section{Introduction}
%Section 1: can do explicit callback to Moron Labs post (we should add that to the standard mss.bib)
% When did urbit backoff on the earth only calls mars? (eyre and iris) Urbit had to run on a VM
In the realm of software, two worlds exist in stark contrast: 
Earth, the familiar domain of established software paradigms, bustling with complex and varied digital ecosystems, 
and Mars, the enigmatic realm of Urbit, a system as pristine and archaic as the Martian landscape itself. 
This article sets forth on an expedition to unravel the mysteries of this Martian software landscape, 
exploring how the intricate processes of terrestrial technology can establish a dialogue with the Martian code.

Unlike Earth, where software is a tapestry of evolution and patchwork, Mars presents a realm of purity. 
It is a world where software exists as a Maxwellian construct, untouched and unaltered by the non-Maxwellian intricacies that characterize Earth’s digital environments. 
Yet, in this isolation lies a profound awareness - an understanding by the Martians that there exist diverse forms of computing with which they must engage. 
Mars, in its wisdom, accommodates these external entities, allowing them to interact in a dialect familiar to its inhabitants: the language of nouns.

In the original explanation of of Urbit, \citeauthor{Yarvin2010}, explicitly says ``The general structure of cross-planet computation is that Earth always calls Mars; Mars never calls Earth. The latter would be quite impossible, since Earth code is non-Maxwellian. There is only one way for a Maxwellian computer to run non-Maxwellian code: in a Maxwellian emulator.'' This however has never been true for Urbit as it was implemented. Urbit has two vanes \texttt{\%eyre} and \texttt{\%iris} that are written as a specific interface to Earth (a web server and curl, respectively). One can even make the argument that Urbit running on a virtual machine specifically breaks this convention. 

Our journey into this uncharted territory begins with a deep dive into the architecture of Urbit, focusing on the foundational layer upon which Arvo, Urbit's kernel, operates.
This exploration will illuminate the operational dynamics and philosophical underpinnings that make Urbit a universe apart. 
Following this, we delve into the intricacies of two pivotal components of Urbit’s ecosystem: 
\texttt{\%khan} and \texttt{\%lick}. 
These vanes serve as distinct conduits between the terrestrial and the Martian, facilitating not just interaction but a nuanced form of control and collaboration with Arvo.
Through this exploration, we aim to demystify the enigma of Urbit and bridge the cosmic gap between Earth’s established software paradigms and the Martian vision of computational purity and simplicity.

\section{Urbit's Unique Landscape}

\subsection{Arvo: The Heart of Martian Computing}
Arvo, also known as Urbit OS, represents a paradigm shift in operating system design. Unlike traditional operating systems that operate on preemptive multitasking and complex event networks, Arvo is a purely functional operating system, characterized by its deterministic nature and compact size. The entire Urbit stack is about 80,000 lines of code, with Arvo itself being only around 2,000 lines. This small codebase is intentional, reflecting a philosophy that system administration complexity is directly proportional to code size.

Arvo's unique architecture avoids what is referred to as "event spaghetti" by maintaining a clear causal chain for every computation. Each chain begins with a Unix I/O event and progresses through a series of steps until the computation's terminal cause. This deterministic nature allows for a high level of predictability and control, a stark contrast to the non-deterministic nature of most Earth-based operating systems.

The kernel's design as a "purely functional operating system" – or more accurately, an "operating function" – underscores its uniqueness. Arvo operates on the principle that the current state is a pure function of its event log, a record of every action ever performed. This determinism is a significant deviation from the norm, where operating systems allow for programmatic alteration of global variables affecting other programs.

Arvo handles nondeterminism in an innovative way. The system, in essence, behaves like a stateful packet transceiver – events are processed, but there's no guarantee of completion, mirroring the behavior of packet dropping in networking. This approach to handling nondeterminism is unique and aligns with Arvo's overall philosophy of simplicity and determinism.

Arvo's determinism is a key feature, stacking it atop a frozen instruction set known as Nock. This concept, though new to operating systems, is not foreign to computing. Similar to how CPU instruction sets like x86-64 are frozen at the chip level, Arvo freezes the instruction set at a higher level, enabling deterministic computation. This high-level determinism contrasts sharply with the non-determinism typically found in Earth's operating systems.

Handling nondeterminism in Arvo is akin to a heuristic decision-making process similar to dropping a packet in networking. This approach views Arvo as a stateful packet transceiver, where the completion of events is not guaranteed, a stark difference from traditional computing models. Additionally, because Arvo runs on a VM, it can obtain nondeterministic information, such as stack traces from infinite loops, through the interpreter beneath it, further enhancing its operational capabilities.

The vision of Urbit, with Arvo at its core, is to transition from developer-hosted web services on multiple foreign servers to self-hosted applications on a personal server. Each architectural decision in Arvo aligns with this vision, emphasizing user ownership and management of data. This approach sets it apart from the conventional model of cloud-based services and centralized data management prevalent in Earth-based systems.

\subsection{Vere: The Substrate of Martian Technology}

Vere, the runtime of Urbit, plays a critical role in actualizing the Martian computing model on Earth-based hardware. As the Nock interpreter written in C, Vere is intricately optimized to run Arvo, ensuring seamless translation of its deterministic operations into practical execution on conventional systems.

The architecture of Vere is bifurcated into two distinct parts: the \texttt{king} and the \texttt{serf}. The \texttt{king} component is responsible for handling vane I/O and managing the effects on Earth. In contrast, the \texttt{serf} operates as the process running the Arvo virtual machine (VM). This division of responsibilities allows for a clear delineation of tasks within the Vere runtime, enhancing its efficiency and robustness.

Communication between the \texttt{king} and \texttt{serf} is achieved through an Inter-Process Communication (IPC) port. This IPC mechanism facilitates a continuous dialogue between the two components, ensuring that the \texttt{king} can effectively manage external interactions while the \texttt{serf} maintains the integrity of the Arvo VM.

A core opcode in Urbit's assembly language (\texttt{nock}) is the \texttt{hint} opcode (opcode 11) this allows Urbit to pass hints to Vere which produces effects on Earth. This bridges the Earth/Mars gap and allows for vanes to interact outside of Mars.

Key to Vere's functionality is how it processes events. Events from the Unix environment are passed to Vere, which then translates and injects them into Arvo. This mechanism demonstrates a profound integration between Vere and Arvo, where Vere possesses direct knowledge of Arvo's state. Notably, Vere can 'scry' or query Arvo's state without altering it, retrieving information as needed. This capability allows Vere to maintain the integrity of Arvo's deterministic model while enabling dynamic interaction with the outside world.

Urbit's use of Unix as its BIOS is a pivotal aspect of its runtime operation. By leveraging the robust, widely-used Unix system, Vere ensures that Urbit can run on a broad range of hardware platforms, effectively making Urbit's Martian technology accessible and operable in Earth's diverse computing environments. This strategic use of Unix not only provides the necessary I/O and optimizations for Arvo but also ensures that Urbit's advanced and unique software architecture can function in tandem with the existing, established hardware and software ecosystems of Earth.

Moreover, Vere's design allows it to inject events into Arvo, a feature crucial for maintaining the fluid communication and interaction between the Urbit system and its underlying hardware and software infrastructure. This bidirectional communication channel ensures that Arvo can respond to external stimuli while remaining true to its functional and deterministic nature. 

Thus, Vere serves as more than just a bridge between Martian technology and Earthly hardware; it is a finely tuned conduit, optimized to uphold and facilitate the unique computational paradigm that Arvo embodies.


\section{\%khan - The High-Level Thread Interface of Urbit}

Khan, as a crucial component of Urbit's architecture, functions as a high-level thread interface, pivotal for managing both internal and external communications within the Urbit ecosystem. It is designed to efficiently handle complex I/O operations, employing an IO monad coupled with an exception monad. This dual-mechanism approach allows Khan to adeptly manage complex IO tasks while effectively handling potential failures.

Khan allows threads to be triggered from outside of Urbit. A thread is a monadic function that takes arguments and produces a result. It may perform input and output while running, so it is not a pure function. A thread's strength is that it can easily perform complex IO operations. It uses what's often called the IO monad (plus the exception monad) to provide a natural framework for IO. A thread's weakness is that it's impermanent and may fail unexpectedly. In most of its intermediate states, it expects only a small number of events (usually one), so if it receives anything it didn't expect, it fails. When code is upgraded, it's impossible to upgrade a running thread, so it fails.

Khan's role in Urbit can be likened to that of a control plane. Its primary function is to execute threads via a Unix Socket and relay the results back. This setup supports various thread-running modes, enhancing the flexibility of Urbit's computing environment and allowing for diverse interaction methods with both internal and external processes.

At its core, Khan's design revolves around the use of 'threads', which are monadic functions designed to handle arguments and produce results. These threads are optimized for I/O operations, leveraging the IO monad for structured and efficient task handling. The exception monad further ensures robust management of potential failures and unexpected events.

Khan's implementation ensures that Urbit serves as a personal server, replacing multiple developer-hosted web services on foreign servers with self-hosted applications on a single personal server. Its design underscores Urbit's commitment to user ownership and data management, setting it apart from the conventional model of cloud-based services and centralized data management prevalent in Earth-based systems.

\section{\texttt{conn.c}} 

\texttt{conn.c}\footnote{\url{https://github.com/urbit/vere/blob/develop/pkg/vere/io/conn.c}} is a driver in Vere. It is a part of the "King" (a.k.a. "Urth") process. It exposes a \texttt{Unix domain socket}\footnote{\url{https://en.wikipedia.org/wiki/Unix_domain_socket}} at \texttt{/path/to/pier/.urb/conn.sock} for sending/receiving data from external processes.

The functionality of conn.c, particularly its ability to dispatch messages and handle various types of requests, is critical for the command of urbit's control plane. It serves as a way for unix processes to receive insights about what is happening on urbit. It also allows for pass commands either \%khan and the injection of raw kernel moves into the system. The implication of this means that whoever is running your urbit runtime controls your urbit. 

From a technical perspective, conn.c accepts newt-encoded \texttt{++jam}med nouns which take the form \texttt{[request-id command arguments]}. 
The newt-encoded format is:

\begin{verbatim}
V.BBBB.JJJJ.JJJJ...
\end{verbatim}

\begin{itemize}
  \item \texttt{V} is the version
  \item \texttt{B} is the jam size in bytes (little endian)
  \item \texttt{J} is the jammed noun (little endian)
\end{itemize}

This structure allows for a variety of commands to be executed, including 

\begin{enumerate}
  \item \texttt{\%ovum} - the injection of raw kernel moves 
  \item \texttt{\%fyrd} - a direct shortcut to Khan commands
  \item \texttt{\%urth} - runtime subcommands like \texttt{\%pack} or \texttt{\%meld}
  \item \texttt{\%peek} - namespace scry requests into Arvo
  \item \texttt{\%peel} - emulated namespace scry requests into Vere
\end{enumerate}

A valid \texttt{conn.c} command produces a newt-encoded jammed noun with type \texttt{[request-id output]}, where:
\begin{itemize}
  \item \texttt{request-id} matches the input \texttt{request-id}
  \item \texttt{output} depends on the \texttt{command}
\end{itemize}

An invalid \texttt{conn.c} command produces a newt-encoded jammed noun with type \texttt{[0 \%bail error-code error-string]}. 

\section{\%lick - The Low Level IPC Interface}

Although also dealing with interprocess communication, \texttt{\%lick} \citep{UIP-101} was designed for a very different scenario than Khan:  to allow external processes, in particular hardware drivers, to intercommunicate with Urbit.  (This breached the Earth/Mars divide.)  Thus \texttt{/sys/vane/lick} focuses on instrumenting a low-level noun interfaces over domain sockets.

\texttt{\%lick} manages IPC ports, and the communication between Urbit applications and POSIX applications via these ports. Other vanes and applications ask \texttt{\%lick} to open an IPC port, notify it when something is connected or disconnected, and transfer data between itself and the Unix application.
Lick works by opening a Unix socket for a particular process, which allows serialized IPC communications.  These involve a jammed noun so the receiving process needs to know how to communicate in nouns.
The IPC ports Lick creates are Unix domain sockets (\texttt{AF\_UNIX} address family) of the \texttt{SOCK\_STREAM} type.
The connexions are made via filepaths in \texttt{.urb/dev} of the pier.

The process on the host OS must therefore strip the first 5 bytes, \texttt{++cue}\footnote{\url{https://docs.urbit.org/language/hoon/reference/stdlib/2p#cue}} the jamfile, check the mark and (most likely) convert the noun into a native data structure.

To understand what \texttt{\%lick} is doing, we need to look at Unix's IPC model briefly.  IPC ("interprocess communication") describes any way that two processes in an operating system's shared context have to communicate with each other.  \texttt{\%lick} focuses on Unix domain sockets\footnote{\url{https://en.wikipedia.org/wiki/Unix_domain_socket}}, which are just communication endpoints\footnote{\url{https://man7.org/linux/man-pages/man7/unix.7.html}}.

For instance, a valid use of \texttt{\%lick} would use cards that look like this:

\begin{verbatim}
++  init  [[%pass / %arvo %l %spin /control]~ this]
::
++  on-arvo
  |=  [=wire =sign-arvo]
  ?+  sign-arvo  (on-arvo:def wire sign-arvo)
      [%lick %soak *]
      ?+  mark.sign-arvo  [~ this]
      ::
        %connect
      ~&  >  "connect"
      :_  this  
          [%pass /spit %arvo %l %spit /control %init area.state]~
      ==  
  ==
::
++  send-state
  |=  =state
  ^-  card:agent:gall
  :*  %pass 
      /spit 
      %arvo %l 
      %spit 
      /control 
      %state 
      [slick:state face.state food.state live.state]
   ==
\end{verbatim}

The vane definition of \texttt{/sys/vane/lick} is even simpler than \texttt{/sys/vane/khan}:  it has no \texttt{++abet} core and primarily communicates to the \texttt{unix-duct} in its state.  The \texttt{+\$owner} is a \texttt{+\$duct} to handle the return \texttt{\%soak}.

Arvo will send three types of \texttt{\%soak}s to an open \texttt{\%lick} port. These \texttt{\%soak}s are \texttt{\%connect} when the first connection is established on the Earth-side IPC port, \texttt{\%disconnect} when the last connection is broken on the Earth-side IPC port, and \texttt{\%error} when an error occurs. \texttt{\%lick} also will send a \texttt{\%disconnect} \texttt{\%soak} to every agent when Vere is started.

Gall needs to wrap \texttt{\%soak} and \texttt{\%spit} to route properly.  See e.g. \texttt{\%++ap-generic-take}.  This lets multiple agents share sockets with the same name, and each agent can have its own folder.

\section{Conclusion}

As Urbit had to deal with the reality of running on Earth software, Developers had to come up with various ways of communicating between these systems. Originally the two vanes \texttt{\%eyre} and \texttt{\%iris} were used as an HTTP client and server, breaking the ``Earth calls Mars; Mars doesn't call Earth'' dynamic. 

Recently, two other vanes, \texttt{\%khan} and \texttt{\%lick} have been developed that hold true to the original idea of communication between Earth and Mars. These vanes force Earth to speak to Mars in a language it can understand, nouns.



\printbibliography
\end{document}
